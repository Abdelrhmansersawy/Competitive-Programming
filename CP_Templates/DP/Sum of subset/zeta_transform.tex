\documentclass[10pt,a4paper]{article}
\usepackage[T1]{fontenc}
\usepackage{amsmath}
\usepackage{amssymb}
\usepackage{geometry}
\usepackage{listings}
\usepackage{xcolor}
\usepackage{hyperref}

\geometry{a4paper, margin=1in}

% Define colors for code listing
\definecolor{codegreen}{rgb}{0,0.6,0}
\definecolor{codegray}{rgb}{0.5,0.5,0.5}
\definecolor{codepurple}{rgb}{0.58,0,0.82}
\definecolor{backcolour}{rgb}{0.95,0.95,0.92}

% Define C++ style for listings
\lstdefinestyle{cppstyle}{
    backgroundcolor=\color{backcolour},   
    commentstyle=\color{codegreen},
    keywordstyle=\color{magenta},
    numberstyle=\tiny\color{codegray},
    stringstyle=\color{codepurple},
    basicstyle=\footnotesize\ttfamily,
    breakatwhitespace=false,         
    breaklines=true,                 
    captionpos=b,                    
    keepspaces=true,                 
    numbers=left,                    
    numbersep=5pt,                  
    showspaces=false,                
    showstringspaces=false,
    showtabs=false,                  
    tabsize=2,
    language=C++
}
\lstset{style=cppstyle}

\hypersetup{
    colorlinks=true,
    linkcolor=blue,
    filecolor=magenta,      
    urlcolor=cyan,
}

\title{\textbf{Complete Notes: Bitwise Convolutions and PIE via SOS DP}}
\author{Competitive Programming Notes}
\date{\today}

\begin{document}
\maketitle
\tableofcontents
\newpage

\section{Introduction}
In competitive programming, we often encounter problems that require computing the convolution of two arrays, $A$ and $B$, of size $2^N$. The general form of a bitwise convolution $C = A * B$ is:
$$ C_k = \sum_{i \oplus j = k} A_i B_j $$
where $\oplus$ can be OR ($|$), AND ($\&$), or XOR ($\wedge$). The technique to solve this efficiently is called Fast Walsh-Hadamard Transform (FWHT). For OR and AND convolutions, the transform used is the \textbf{Zeta Transform}, which is implemented using Sum over Subsets (SOS) DP.

\section{OR Convolution}
The OR convolution of two arrays $A$ and $B$ is defined as:
$$ C_k = (A *_{OR} B)_k = \sum_{i | j = k} A_i B_j $$

\subsection{The Zeta Transform}
To compute this efficiently, we use the \textbf{Zeta Transform}, also known as the Sum over Subsets (SOS) transform. For an array $A$, its Zeta Transform $\hat{A}$ is defined as:
$$ \hat{A}_S = \sum_{T \subseteq S} A_T $$
The key idea is that the Zeta transform converts the complex OR convolution into a simple pointwise product. Let $\hat{A}$ and $\hat{B}$ be the Zeta transforms of $A$ and $B$. Let $\hat{C}$ be their pointwise product: $\hat{C}_S = \hat{A}_S \cdot \hat{B}_S$. Then, $C$ is the OR convolution of $A$ and $B$.

\subsection{Algorithm}
\begin{enumerate}
    \item Compute the Zeta transforms: $\hat{A} = \text{Zeta}(A)$ and $\hat{B} = \text{Zeta}(B)$.
    \item Compute the pointwise product: $\hat{C}_S = \hat{A}_S \cdot \hat{B}_S$ for all $S$.
    \item Compute the inverse Zeta transform to get the result: $C = \text{InverseZeta}(\hat{C})$.
\end{enumerate}

\subsection{Implementation}
\begin{lstlisting}
#include <bits/stdc++.h>
using namespace std;

// Computes the OR convolution of two vectors
vector<long long> or_conv(vector<long long> a, vector<long long> b) {
    int sz = a.size();
    int n = __builtin_ctz(sz);

    auto sos_dp = [&](vector<long long>& v) {
        for (int i = 0; i < n; ++i) {
            for (int mask = 0; mask < sz; ++mask) {
                if (mask & (1 << i)) v[mask] += v[mask ^ (1 << i)];
            }
        }
    };

    auto inv_sos_dp = [&](vector<long long>& v) {
        for (int i = 0; i < n; ++i) {
            for (int mask = 0; mask < sz; ++mask) {
                if (mask & (1 << i)) v[mask] -= v[mask ^ (1 << i)];
            }
        }
    };

    sos_dp(a);
    sos_dp(b);

    vector<long long> c(sz);
    for (int i = 0; i < sz; ++i) c[i] = a[i] * b[i];

    inv_sos_dp(c);
    return c;
}
\end{lstlisting}

\section{AND Convolution}
The AND convolution is defined similarly:
$$ C_k = (A *_{AND} B)_k = \sum_{i \& j = k} A_i B_j $$
For this, we use the \textbf{Dual Zeta Transform}, or Sum over Supersets transform.
$$ \hat{A}_S = \sum_{S \subseteq T} A_T $$
The algorithm remains the same: transform $A$ and $B$, compute the pointwise product, and apply the inverse transform.

\subsection{Implementation}
\begin{lstlisting}
#include <bits/stdc++.h>
using namespace std;

// Computes the AND convolution of two vectors
vector<long long> and_conv(vector<long long> a, vector<long long> b) {
    int sz = a.size();
    int n = __builtin_ctz(sz);

    auto dual_sos_dp = [&](vector<long long>& v) {
        for (int i = 0; i < n; ++i) {
            for (int mask = sz - 1; mask >= 0; --mask) {
                if (mask & (1 << i)) v[mask ^ (1 << i)] += v[mask];
            }
        }
    };

    auto inv_dual_sos_dp = [&](vector<long long>& v) {
        for (int i = 0; i < n; ++i) {
            for (int mask = sz - 1; mask >= 0; --mask) {
                if (mask & (1 << i)) v[mask ^ (1 << i)] -= v[mask];
            }
        }
    };

    dual_sos_dp(a);
    dual_sos_dp(b);

    vector<long long> c(sz);
    for (int i = 0; i < sz; ++i) c[i] = a[i] * b[i];

    inv_dual_sos_dp(c);
    return c;
}
\end{lstlisting}

\section{PIE: From Unions to Intersections}
A powerful application of SOS DP is to compute values related by the Principle of Inclusion-Exclusion. This section covers converting information about the sizes of \textbf{unions} of sets into information about the sizes of their \textbf{intersections}. 
\begin{itemize}
    \item \textbf{Input Meaning:} The input array \texttt{U[mask]} is $|\bigcup_{i \in \text{mask}} S_i|$.
    \item \textbf{Output Meaning:} The function calculates \texttt{I[mask]} = $|\bigcap_{i \in \text{mask}} S_i|$.
\end{itemize}
The PIE formula states:
$$ |\bigcap_{i \in S} A_i| = \sum_{\emptyset \neq T \subseteq S} (-1)^{|T|-1} |\bigcup_{j \in T} A_j| $$
If we let $I_{mask} = |\bigcap_{i \in \text{mask}} S_i|$ and $U_{mask} = |\bigcup_{i \in \text{mask}} S_i|$, and define an intermediate array $F_{sub} = (-1)^{|sub|-1} U_{sub}$, the formula becomes $I_{mask} = \sum_{sub \subseteq mask} F_{sub}$. This is precisely the Zeta Transform of $F$.

\subsection{Implementation}
\begin{lstlisting}
#include <bits/stdc++.h>
using namespace std;

// Converts union sizes to intersection sizes.
vector<long long> unions_to_intersections(vector<long long> u) {
    int sz = u.size();
    int n = __builtin_ctz(sz);
    
    vector<long long> res(sz);

    // Apply PIE coefficients to create an intermediate array.
    for (int mask = 1; mask < sz; ++mask) {
        if (__builtin_popcount(mask) % 2 == 1) {
            res[mask] = u[mask];
        } else {
            res[mask] = -u[mask];
        }
    }

    // Apply Zeta Transform (Sum over Subsets) to get intersection sizes.
    for (int i = 0; i < n; ++i) {
        for (int mask = 0; mask < sz; ++mask) {
            if (mask & (1 << i)) {
                res[mask] += res[mask ^ (1 << i)];
            }
        }
    }
    return res;
}
\end{lstlisting}

\section{Trick: Counting Exact Memberships}
After computing intersection sizes, we can find the number of elements that belong to \textbf{exactly} the sets indexed by a mask. Let $E_{mask}$ be this count (i.e., elements in $S_i$ for all $i \in mask$ and not in $S_j$ for all $j \notin mask$).

An element that belongs to exactly the sets in a superset $T \supseteq S$ is counted in the intersection $I_S$. Therefore, the intersection size is the sum of exact counts of all its supersets:
$$ I_S = \sum_{T \supseteq S} E_T $$
This is the definition of the Dual Zeta Transform (Sum over Supersets). To find the exact counts $E$ from the intersection sizes $I$, we simply apply the \textbf{Inverse Dual Zeta Transform} to the array of intersection sizes.

\subsection{Implementation}
\begin{lstlisting}
#include <bits/stdc++.h>
using namespace std;

// Converts intersection sizes to the count of elements in EXACTLY the sets of a mask.
vector<long long> intersections_to_exact_counts(vector<long long> intersections) {
    int sz = intersections.size();
    int n = __builtin_ctz(sz);
    
    vector<long long> res = intersections;

    // Apply Inverse Dual Zeta Transform (Inverse Sum over Supersets).
    for (int i = 0; i < n; ++i) {
        for (int mask = sz - 1; mask >= 0; --mask) {
            if (mask & (1 << i)) {
                res[mask ^ (1 << i)] -= res[mask];
            }
        }
    }
    return res;
}
\end{lstlisting}

\end{document}
