\documentclass{article}
\usepackage{amsmath}
\usepackage{amsfonts}
\usepackage{amssymb}

% Optional: Adjust page margins for better layout
\usepackage[margin=1in]{geometry}

\begin{document}

\section*{Contribution of a Single Node}

Given an array of length $n$, where some elements are marked as \textbf{infected}, the problem asks for the following sum:
\[
    S = \sum_{\text{all subsegments } x} \frac{\text{number of infected elements in } x}{\text{length of } x}.
\]

A key insight in combinatorics is to change the order of summation. Instead of summing over subsegments, we can sum over each infected element. The formula can be rewritten as:
\[
    S = \sum_{\substack{\text{infected elements } w\\\text{at position } i}}
         \Biggl(\sum_{\substack{\text{subsegments } x\\\text{that contain position } i}}
               \frac{1}{\text{length of } x}\Biggr).
\]

\bigskip\hrule\bigskip

\section*{Computing Each Element’s Contribution in $O(1)$ with Modular Inverses}

For an array of $n$ elements labeled $1$ through $n$, and a fixed infected element $w$ at position $i$, its contribution is:
\[
    \mathrm{Contrib}(i)
    = \sum_{L=1}^{i}\sum_{R=i}^{n}\frac{1}{R-L+1}
    = \sum_{d=0}^{i-1}\sum_{d'=0}^{n-i}\frac{1}{d+d'+1}.
\]
Using precomputed \textbf{modular inverses} instead of floating-point division, we get the same identity:
\[
    \sum_{L=1}^{i}\sum_{R=i}^{n}\frac{1}{R-L+1}
    =
    SH_{n-1} - SH_{i-1} - SH_{n-i}
    \pmod{M},
\]
where all operations are modulo $M$ (e.g.\ $10^9+7$) and
\[
    H_n = \sum_{t=1}^n \tfrac{1}{t} = \sum_{t=1}^n \mathrm{inv}[t],
    \qquad
    SH_n = \sum_{t=1}^n H_t,
\]
with $\mathrm{inv}[t]$ the modular inverse of $t \bmod M$.

\subsection*{Total}
\[
    S = \sum_{\text{infected }w}\mathrm{Contrib}(w) \pmod{M}.
\]
This yields an $O(n + I)$ algorithm using purely integer arithmetic and modular inverses, where $I$ is the number of infected elements.

\end{document}
