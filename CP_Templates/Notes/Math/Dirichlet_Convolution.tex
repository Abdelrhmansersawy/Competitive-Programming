\documentclass[12pt]{article}
\usepackage{amsmath, amssymb}
\usepackage{geometry}
\usepackage{fancyhdr}
\usepackage{mathtools}
\usepackage{hyperref}
\geometry{a4paper, margin=1in}
\setlength{\parskip}{1em}
\pagestyle{fancy}
\fancyhf{}
\rhead{Primitive Strings and Möbius Inversion}
\lhead{Math Notes}
\rfoot{\thepage}

\title{Primitive Strings and Möbius Inversion}
\author{}
\date{}

\begin{document}

\maketitle

\section*{Dirichlet Convolution Identity}

We start with a well-known identity from number theory:

\[
\text{If } f(x) = \sum_{d \mid x} g(d),
\text{ then (by Möbius inversion): }
g(x) = \sum_{d \mid x} \mu(d) \cdot f\left(\frac{x}{d}\right)
\]

Here, $\mu(d)$ is the Möbius function.

\section*{Application: Counting Primitive Strings}

Let:
\begin{itemize}
  \item $A$ be the size of the alphabet,
  \item $P(x)$ be the number of primitive strings of length $x$,
  \item $A^x$ be the total number of strings of length $x$ over the alphabet.
\end{itemize}

\section*{Inclusion-Exclusion (Recursive Definition)}

We can write:

\[
P(x) = A^x - \sum_{\substack{d \mid x \\ d < x}} P(d)
\]

This makes sense because:
\begin{itemize}
  \item There are $A^x$ total strings of length $x$,
  \item Some are periodic: they are repetitions of primitive strings of smaller lengths.
\end{itemize}

We can rearrange the above equation as:

\[
A^x = \sum_{d \mid x} P(d) \tag{1}
\]

This states that every string of length $x$ can be uniquely formed by repeating a primitive string of some length $d \mid x$.

\section*{Deriving $P(x)$ Using Möbius Inversion}

Let:
\[
f(x) = A^x, \quad g(d) = P(d)
\]

Then Equation (1) becomes:

\[
f(x) = \sum_{d \mid x} g(d)
\]

Applying Möbius inversion:

\[
g(x) = \sum_{d \mid x} \mu(d) \cdot f\left( \frac{x}{d} \right)
\]

Substituting back:
\[
P(x) = \sum_{d \mid x} \mu(d) \cdot A^{x/d}
\]

\[
\boxed{
P(x) = \sum_{d \mid x} \mu(d) \cdot A^{x/d}
}
\]

\section*{Summing Primitive Strings up to Length $k$}

To compute the total number of primitive strings of length $\le k$:

\[
\sum_{x=1}^{k} P(x) = \sum_{x=1}^{k} \sum_{d \mid x} \mu(d) \cdot A^{x/d}
\]

This evaluates each $P(x)$ separately, then adds them up.  
However, we can rearrange the summation for better efficiency.

\section*{Floor-Based Reordering (Summation Swap)}

Start again from:

\[
\sum_{x=1}^{k} P(x) = \sum_{x=1}^{k} \sum_{d \mid x} \mu(d) \cdot A^{x/d}
\]

Now swap the summation order.  
Instead of fixing $x$ and looping over $d \mid x$, we fix $d$ and sum over all multiples $x = d \cdot j \le k$:

\[
\sum_{x=1}^{k} P(x) = \sum_{d=1}^{k} \mu(d) \cdot \sum_{j=1}^{\left\lfloor \frac{k}{d} \right\rfloor} A^j
\]

\subsection*{Optional Variable Renaming}

Renaming $d \rightarrow x$ and keeping $j$ the same:

\[
\sum_{x=1}^{k} P(x) = \sum_{x=1}^{k} \sum_{j=1}^{\left\lfloor \frac{k}{x} \right\rfloor} \mu(j) \cdot A^j
\]

\section*{Final Form}

\[
\boxed{
\sum_{x=1}^{k} P(x) = \sum_{d=1}^{k} \mu(d) \cdot \sum_{j=1}^{\left\lfloor \frac{k}{d} \right\rfloor} A^j
}
\]

This form is especially useful for implementation and number-theoretic optimization.

\section*{Efficient Evaluation in $O(k)$ Time}

Each inner sum is a geometric series:

\[
\sum_{j=1}^{n} A^j = A \cdot \frac{A^n - 1}{A - 1}
\]

where $n = \left\lfloor \frac{k}{d} \right\rfloor$.

So the full expression becomes:

\[
\sum_{x=1}^{k} P(x) = \sum_{d=1}^{k} \mu(d) \cdot \left( A \cdot \frac{A^{\left\lfloor \frac{k}{d} \right\rfloor} - 1}{A - 1} \right)
\]

Since we compute this in constant time per $d$, the total time complexity is:

\[
\boxed{O(k)}
\]

\end{document}
