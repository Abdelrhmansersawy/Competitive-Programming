\documentclass[12pt]{article}
\usepackage{amsmath,amssymb}
\usepackage{geometry}
\geometry{margin=1in}
\setlength{\parskip}{1em}
\setlength{\parindent}{0pt}

\title{Expected Value Laws and Generating Functions}
\author{}
\date{}

\begin{document}

\maketitle

\section*{1. Expected Value via States}

For a process with multiple outcomes or states:

\[
\mathbb{E}[\text{cur\_state}] = 1 + \sum_i p_i \cdot \mathbb{E}[S_i] = \sum_i p_i \cdot \text{val}_i
\]

\begin{itemize}
  \item \( S_i \): the $i$-th resulting state.
  \item \( p_i \): the probability of moving to state \( S_i \).
  \item \( \text{val}_i \): known expected value for state \( S_i \), assuming recursive breakdown.
\end{itemize}

\section*{2. Linearity of Expectation}

\[
\mathbb{E}[X + Y] = \mathbb{E}[X] + \mathbb{E}[Y]
\]

\textbf{This holds unconditionally}, even if \( X \) and \( Y \) are dependent.

\section*{3. Expectation of Product of Independent Variables}

\[
\mathbb{E}[XY] = \mathbb{E}[X] \cdot \mathbb{E}[Y]
\]

\textbf{This is valid only when} \( X \) and \( Y \) are independent.

\section*{4. Generating Functions}

Generating functions compactly encode a sequence of probabilities:

\[
G(s) = P_0 + P_1 s + P_2 s^2 + P_3 s^3 + \cdots
\]

\begin{itemize}
  \item \( P_i \): probability that a random variable equals \( i \).
  \item \( G(s) \): probability generating function (PGF).
\end{itemize}

\section*{5. Expected Value from Generating Function}

Given the generating function \( G(s) \), the expected value is:

\[
\mathbb{E}[X] = G'(1)
\quad \text{where} \quad
G'(s) = \frac{dG(s)}{ds}
\]

You differentiate \( G(s) \) and evaluate at \( s = 1 \) to get the mean.

\section*{6. Tail Sum Formula for Expected Value}

For a \textbf{non-negative integer-valued} random variable \( X \):

\[
\mathbb{E}[X] = \sum_{i=1}^{\text{max possible}} \mathbb{P}(X \geq i)
\]

\textbf{Explanation:}
\begin{itemize}
  \item Instead of computing \( i \cdot \mathbb{P}(X = i) \), this sums the \emph{tail probabilities} \( \mathbb{P}(X \ge i) \).
  \item This is known as the \textbf{tail sum formula}.
\end{itemize}

\textbf{When to Use:}
\begin{itemize}
  \item When \( X \in \mathbb{Z}_{\ge 0} \) (i.e., non-negative integers).
  \item When tail probabilities \( \mathbb{P}(X \ge i) \) are easy to compute or are directly given.
\end{itemize}

\textbf{Example (Conceptual):}

If a process continues with probability \( p \) and stops otherwise, the expected number of steps can be computed by summing \( \mathbb{P}(X \ge i) \) over all \( i \ge 1 \).

\end{document}
