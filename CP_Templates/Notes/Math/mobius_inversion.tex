\documentclass[12pt]{article}
\usepackage{amsmath,amssymb}
\usepackage{geometry}
\geometry{margin=1in}
\setlength{\parskip}{1em}
\setlength{\parindent}{0pt}

\title{Möbius Inversion and Summation of $g(x)$}
\author{}
\date{}

\begin{document}

\maketitle

\section*{Möbius Inversion and Summation of \texorpdfstring{$g(x)$}{g(x)}}

Suppose you're given:

\[
f(x) = \sum_{d \mid x} g(d)
\]

and you want to recover $g(x)$.  
Using Möbius inversion:

\[
\boxed{
g(x) = \sum_{d \mid x} \mu(d) \cdot f\left( \frac{x}{d} \right)
}
\]

Now, to compute the \textbf{sum of $g(x)$ from $x = 1$ to $k$}, apply the Möbius formula over the range:

\[
\sum_{x = 1}^{k} g(x) = \sum_{x = 1}^{k} \sum_{d \mid x} \mu(d) \cdot f\left( \frac{x}{d} \right)
\]

We now \textbf{swap the order of summation}.  
Let \( x = d \cdot j \), so \( j = \frac{x}{d} \), and note that \( x \le k \Rightarrow j \le \left\lfloor \frac{k}{d} \right\rfloor \). This gives:

\[
\boxed{
\sum_{x=1}^{k} g(x) = \sum_{d=1}^{k} \mu(d) \cdot \sum_{j=1}^{\left\lfloor \frac{k}{d} \right\rfloor} f(j)
}
\]

This form is often more efficient to compute, especially when $f(j)$ is memoized or precomputed.

We can also rewrite it by renaming variables ($d \to x$), which yields:

\[
\boxed{
\sum_{x=1}^{k} g(x) = \sum_{x=1}^{k} \sum_{j=1}^{\left\lfloor \frac{k}{x} \right\rfloor} \mu(j) \cdot f(j)
}
\]

\section*{Final Summary}

If:

\[
f(x) = \sum_{d \mid x} g(d)
\quad \text{then} \quad
g(x) = \sum_{d \mid x} \mu(d) \cdot f\left(\frac{x}{d}\right)
\]

Then summing over $x$ gives:

\[
\sum_{x = 1}^{k} g(x)
= \sum_{d=1}^{k} \mu(d) \cdot \sum_{j=1}^{\left\lfloor \frac{k}{d} \right\rfloor} f(j)
= \sum_{x=1}^{k} \sum_{j=1}^{\left\lfloor \frac{k}{x} \right\rfloor} \mu(j) \cdot f(j)
\]

All expressions above are \textbf{equivalent and valid}.  
The only incorrect form would be:

\[
\sum_{x=1}^{k} g(x) = \sum_{d \mid x} \mu(d) \cdot f\left( \frac{x}{d} \right)
\]

because it references $x$ without a proper summation over $x$ on the right-hand side.

\end{document}
